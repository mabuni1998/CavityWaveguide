In conclusion, we have in this Msc.~thesis successfully implemented the numerical framework \href{https://github.com/qojulia/WaveguideQED.jl}{WaveguideQED.jl} for simulating waveguide Quantum Electrodynamics (QED) problems. The tool is built on top of the QuantumOptics.jl library and maintains the excellent user experience offered by the original library. Through convergence studies, we have demonstrated the accuracy of the framework, ensuring that it produces correct results. Additionally, to enhance performance, we have introduced LazyOperators and discussed efficient representations of the waveguide state. 

By utilizing \href{https://github.com/qojulia/WaveguideQED.jl}{WaveguideQED.jl}, we have studied the scattering of two-photon pulses on a cavity and an emitter, reproducing recent experimental results. We demonstrate the non-linear capabilities of a single-emitter by showing that a scattered two-photon pulse has signatures of entanglement across time due to stimulated emission. On the contrary, scattering a two-photon pulse on a cavity does not lead to entanglement across time, only producing a product state, revealing that the photons effectively did not interact. Furthermore, we have described internal waveguide coupling and shown how to successfully model a defect in a waveguide leading to partial reflection of one waveguide mode into another. We show that such a defect changes the emission frequency and rate of a coupled emitter due to variations in the local density of optical states. We also predict the emergence of Fano resonances under certain parameter conditions of this scattering element. Finally, we demonstrated the framework's capability to treat non-Markovian dynamics by considering a semi-infinite waveguide with a mirror in one end. The emitted light is here fed back into the system leading to a complex feedback loop. We observe excitation trapping for a mirror phase of $\phi = \pi$, confirming other theoretical results. We extend those studies by considering how single- and two-photon pulses scatter on the system, where we no longer observe excitation trapping.

\

In essence, we have presented a versatile simulation tool that offers researchers in the field of quantum optics a simple but powerful approach. The software interface is intuitive and will be familiar to those already experienced with quantum optics software such as Qutip in Python, Quantum Toolbox in MATLAB, or QuantumOptics.j in Julia. By describing the total state of the waveguide without relying on matrix-product states, our simulation tool reduces the knowledge-barrier-to-entry, providing a transparent approach that is less of a black box. This allows for the exploration of non-Markovian dynamics while still maintaining an accessible framework. It is important to note that this comes with limitations, and specifically, it is, as of now, only possible to describe up to two photons simultaneously. Although it is, in principle, possible to extend the capabilities to allow for a larger number of photons, it comes at a significant numerical cost. In such cases, employing matrix product states might still be a better solution at the cost of complexity. 

\

Looking ahead, future work involves investigating the impact of non-Markovian dynamics on the preparation and alteration of exotic photonic states.  For example, 2D photonic cluster states from waveguides with feedback have been proposed \cite{Pichler2016PhotonicFeedback}. Recently, it was also studied how a single emitter can perform a two-photon splitting operation \cite{Lund2023PerfectPulse}. It is here interesting to consider if any advantage can be gained by introducing non-markovian dynamics or whether other schemes can be derived from such configurations. Furthermore, the optimization of schemes performing nonlinear operations on the photonic state is also a promising direction. Indeed, the framework is based on an analytical approach, which has already been employed to optimize C-phase gates by considering different nonlinearities \cite{Heuck2020Photon-photonCavities,Heuck2020Controlled-PhaseNonlinearities,Krastanov2022Controlled-phaseEmitter}. It is here possible that by using the numerical framework one can extend the analysis to more complex systems. Finally, the approach could also help to solve few-photon transport problems in more complex waveguide QED systems. Again, previous analysis has been performed using analytically derived scattering matrices \cite{Joanesarson2020Few-photonGeometries}, and with a numerical approach, it is possible to imagine the study of more complex systems.
