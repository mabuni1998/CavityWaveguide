\section{Rotating frame \label{app:rotating}}
In the quantum optical collision formalisms, the Hamiltonian is transformed into an interaction picture to introduce the notion of time bins. We consider the general Hamiltonian for $N$ waveguides interacting with a local quantum system $H_s$ with the annihilation operator $a$ (for an emitter system this would instead be $\sigma$) \cite{Xu2016FanoTransport}:
\begin{equation}
\begin{aligned}
H= H_{\mathrm{s}}+ \sum_{i=1}^N \int d \nu \hbar \nu w_{i}^{\dagger}(\nu) w_{i}(\nu)+\sum_{i=1}^N \hbar \sqrt{\gamma_i} \int \frac{d \nu}{\sqrt{2 \pi}}\left(w_{i}^{\dagger}(\nu) a+a^{\dagger} w_{i}(\nu)\right) +\sum_{i \neq j} \hbar 
 V_{i j} \int \frac{d \nu}{\sqrt{2 \pi}} \int \frac{d  \nu^{\prime}}{\sqrt{2 \pi}} w_{i}^{\dagger}(\nu) w_{j}(\nu^\prime),
\end{aligned}
\end{equation}
where $w_{i}(\nu)$ is the annihilation operator for a mode in waveguide $i$ with frequency $\nu$, $\gamma_i$ is the coupling between the local system and waveguide $i$, and $V_{i,j} =V_{j,i}$ is the coupling between waveguides. If we move into an interaction picture with regards to $H_0 =\hbar \omega_s a^\dagger a + \hbar 
 \sum_{i=1}^N \int d \nu \nu w_{i}^{\dagger}(\nu) w_{i}(\nu)$, the waveguide operators transform as:
\begin{equation}
    \mathrm{e}^{i H_0 t/\hbar } w_i(\nu) \mathrm{e}^{-i H_0 t/\hbar } = w_i(\nu) - i t \nu w_i(\nu) + 
    \frac{(-it\nu)^2}{2!} w_i(\nu) + \cdots = w_i(\nu) \mathrm{e}^{-i\nu t}
\end{equation}
and the system operators likewise as:
\begin{equation}
    \mathrm{e}^{i H_0 t/\hbar } a \mathrm{e}^{-i H_0 t/\hbar } = a \mathrm{e}^{-i \omega_s t}
\end{equation}
In the interaction picture, the transformed Hamiltonian $\Tilde{H}$ then becomes \cite{Breuer2006TheSystems}:
\begin{align}
\Tilde{H} = \mathrm{e}^{i H_0 t/\hbar } H \mathrm{e}^{-i H_0 t/\hbar } - H_0 &= H_{\mathrm{s}} - \hbar \omega_s a^\dagger a +\sum_{i=1}^N \hbar \sqrt{\gamma_i} \int \frac{d \nu}{\sqrt{2 \pi}}\left(w_{i}^{\dagger}(\nu) \mathrm{e}^{i(\nu - \omega_s) t} a+a^{\dagger} w_{i}(\nu) \mathrm{e}^{-i(\nu - \omega_s) t}\right)  \\
&+ \sum_{i \neq j} \hbar V_{i j} \int \frac{d \nu}{\sqrt{2 \pi}} \int \frac{d  \nu^{\prime}}{\sqrt{2 \pi}} w_{i}^{\dagger}(\nu) \mathrm{e}^{i\nu t} w_{j}(\nu^\prime) \mathrm{e}^{-i\nu^\prime t},
\end{align}
If we introduce the Fourier transformed waveguide operator $w_i(t) =\int \frac{d \nu}{\sqrt{2 \pi}} w_{i}(\nu) \mathrm{e}^{-i(\nu - \omega_s) t}$ the above simplifies to:
\begin{align}
\Tilde{H} = H_{\mathrm{s}}-\hbar \omega_s a^\dagger a+\sum_{i=1}^N \hbar \sqrt{\gamma_i} \left(w_{i}^{\dagger}(t) a+a^{\dagger} w_{i}(t)\right) + \sum_{i \neq j} \hbar V_{i j} w_{i}^{\dagger}(t) w_{j}(t), \label{eq:general_h}
\end{align}
Notice that for the waveguide interaction term, we inserted $\mathrm{e}^{-i \omega_s t}\mathrm{e}^{i \omega_s t} = 1$. Also, we see that the waveguide pulse has a frequency centered around $\omega_s$. If we instead wanted to describe a waveguide pulse centered around $\omega_w$ and a system with frequency $\omega_s$ we would transform using $H_0 =\hbar \omega_w a^\dagger a + \sum_{i=1}^N \int d \nu \hbar \nu w_{i}^{\dagger}(\nu) w_{i}(\nu)$ which would lead to a term like $\delta a^\dagger a$ in the system Hamiltonian $H_s$, where $\delta = \omega_s - \omega_w$.

In the collision framework, we then interpret equation \eqref{eq:general_h} as describing an interaction with photon bins. This can be understood by considering how the unitary evolution operator evolving from a time $t_{n-1}$ to a time $t_n$ where $t_n - t_{n-1} = \Delta t$ looks like:
\begin{align}
    U(t_{n-1},t_n) &= \exp \left (-i \int_{t_{n-1}}^{t_n} \mathrm{d}t^\prime H_{\mathrm{s}}-\omega_s a^\dagger a+\sum_{i=1}^N \sqrt{\gamma_i} \left(w_{i}^{\dagger}(t^\prime) a+a^{\dagger} w_{i}(t^\prime)\right) + \sum_{i \neq j} V_{i j} w_{i}^{\dagger}(t^\prime) w_{j}(t^\prime) \right ) \\
    &
\end{align}
If we now introduce a discretized operator $w_{n,i} = \frac{1}{\sqrt{\Delta t}} \int_{t_{n-1}}^{t_n} \mathrm{d}t^\prime w_{i}(t^\prime)$ that satisfies the commutation relation: $\comm{w_{n,i}}{w_{n^\prime,j}^\dagger} =  \frac{1}{\Delta t} \int_{t_{n-1}}^{t_n} \mathrm{d}t^\prime \int_{t_{n^\prime-1}}^{t_n^\prime} \mathrm{d}t^{\prime \prime} \comm{w_i(t^\prime)}{w_j(t^{\prime \prime})} =  \frac{1}{\Delta t} \int_{t_{n-1}}^{t_n} \mathrm{d}t^\prime \int_{t_{n^\prime-1}}^{t_n^\prime} \mathrm{d}t^{\prime \prime} \delta(t^\prime - t^{\prime \prime}) \delta_{i,j} = \delta_{n,n^\prime} \delta_{i,j}$, we can write the evolution operator as (we assume that the system Hamiltonian is time-independent for the sake of simplicity, but it is by no means a requirement):
\begin{align}
    U(t_{n-1},t_n) &= \exp \left (-i \Delta t H_{\mathrm{s}}-\Delta t \omega_s a^\dagger a\mathrm{d}t^\prime + \sqrt{\Delta t} \sum_{i=1}^N \sqrt{\gamma_i} \left(w_{n,i}^{\dagger} a+a^{\dagger} w_{n,i} \right) + \Delta t \sum_{i \neq j} V_{i j} w_{n,i}^{\dagger} w_{n,j} \right ) \\
\end{align}
Notice that if we take the $\Delta t$ outside, we get:
\begin{align}
    U(t_{n-1},t_n) &= \exp \left (-i \Delta t \left [ H_{\mathrm{s}}- \omega_s a^\dagger a\mathrm{d}t^\prime + \sum_{i=1}^N \sqrt{\gamma_i/\Delta t} \left(w_{n,i}^{\dagger} a+a^{\dagger} w_{n,i} \right) + \sum_{i \neq j} V_{i j} w_{n,i}^{\dagger} w_{n,j} \right ] \right ) \\
\end{align}
We thus have a rescaled Hamiltonian:
\begin{equation}
    H_{\mathrm{s}}- \hbar \omega_s a^\dagger a  + \sum_{i=1}^N \hbar 
 \sqrt{\gamma_i/\Delta t} \left(w_{n,i}^{\dagger} a+a^{\dagger} w_{n,i} \right) + \sum_{i \neq j}\hbar  V_{i j} w_{n,i}^{\dagger} w_{n,j}
\end{equation}
that leads to a constant evolution within each time bin. We could also have arrived at this Hamiltonian by introducing the discretized operators as  a transformation of the continuous: $w(k \Delta t) \rightarrow  \frac{w_k}{\sqrt{\Delta t}}$.

If we rescale $V_{ij}$, we see that we recover the beamsplitter transformation. One caveat, however, applies, since we made the following approximation when we discretized the interaction:
\begin{equation}
    \int_{t_{n-1}}^{t_n} w_i^\dagger(t^\prime) w_j(t^\prime) \approx \int_{t_{n-1}}^{t_n} w_i^\dagger(t^\prime) \int_{t_{n-1}}^{t_n} w_j(t^\prime) = \Delta t w_{n,i}^\dagger w_{n,j}
\end{equation}
whether this approximation is accurate is not obvious, but we do see that we recover the unitary transformation of a beamsplitter with it. In sec. \ref{sec:waveguideIO} and \ref{sec:inferringwaveguide}, we discuss the possible problems with making this approximation and possible solutions to it. 





\section{Analytical solution to single-photon scattering \label{app:analytical}}

In sec. \ref{sec:onephoton_scat}, we derive the following differential equation for the scattering of a single photon on a cavity:  
\begin{equation}
    \frac{d \psi_1(t)}{d t}  = -(i \delta_a +\frac{\gamma}{2}) \psi_1(t) + \sqrt{\gamma} \xi^{(1)}(t)
\end{equation}
The general solution for this differential equation is given as \cite{Arfken2005MathematicalPhysicists}:
\begin{equation}
\psi_1(t) =  \mathrm{e}^{-(i\delta_a+\frac{\gamma}{2}) t} \left [ \psi_1(0) + \int_{0}^t \mathrm{e}^{-(i\delta_a+\frac{\gamma}{2}) t} \xi(s)  ds \right ]
\end{equation}
In sec. \ref{sec:onephoton_numerical}, we consider a Gaussian input state with $\xi^{(1)}(t) = \sqrt{\frac{2}{\sigma}} \left(\frac{\log(2)}{\pi}\right)^{1/4} \exp\left(-\frac{2\log(2)(t-t_0)^2}{\sigma^2}\right) $ and $\psi_1(0)=0$.
Inserting, we thus have the integral:
%\sqrt{\frac{2\gamma}{\sigma}}\left(\frac{\log 2}{\pi}\right)^{\frac{1}{4}}\exp\left[-\frac{2\log^2 2}{\sigma^2}(s-t_0)^2\right]
\begin{equation}
\psi_1(t) = \sqrt{\frac{2}{\sigma}}\left(\frac{\log(2)}{\pi}\right)^{\frac{1}{4}}\int_{0}^t e^{-(i\delta_a+\frac{\gamma}{2})t}\exp\left[-\frac{2\log(2)}{\sigma^2}(s-t_0)^2\right] ds
\end{equation}

Using,
\begin{equation}
    \int_{0}^{t} e^{-a(s-t_0)^2+b s}ds = \frac{\sqrt{\pi} e^{\frac{b^2}{4a}+bt_0}}{2\sqrt{a}}\left(\text{erf}\left(\frac{2a(t-t_0)-b}{2\sqrt{a}}\right) + \text{erf}\left(\frac{2at_0+b}{2\sqrt{a}}\right)\right)
\end{equation}
together with the input-output relation $\xi_{out}(t) = \xi_{in}(t) - \sqrt{\gamma}\psi(t)$ we get:
\begin{equation}
    \xi^{(1)}_\mathrm{out}(t) =  \xi_{in}^{(1)}(t) - \sqrt{\gamma} \frac{\sqrt{\pi} e^{\frac{b^2}{4a}+bt_0}}{2\sqrt{a}}\left(\text{erf}\left(\frac{2a(t-t_0)-b}{2\sqrt{a}}\right) + \text{erf}\left(\frac{2at_0+b}{2\sqrt{a}}\right)\right)
\end{equation}
with $a = 2 \log(2)/\sigma^2$ and $b = i \delta + \gamma/2$. Which is also the result in eq. \eqref{eq:onephoton_analytical}.



\section{Waveguide Interaction \label{app:waveguideinteraction}}
In chapter \ref{ch3}, we consider how multiple waveguides interact, and in this appendix, we elaborate on some of the derivations. We consider two waveguides labeled $a$ and $b$, which interact through the following Hamiltonian:
\begin{equation}
     H_{int}(t) = \sum_k f_k(t) H_k = \sum_k f_k(t) \frac{\hbar V}{\Delta t}\left(w_{k, a}^{\dagger} w_{k, b}+w_{k, b}^{\dagger} w_{k, a}\right)= \sum_k f_k(t)\frac{\hbar V}{\Delta t} O_k
\end{equation}
where $f_k(t)$ is as defined as in eq. \eqref{eq:fk} and $O_k = \left(w_{k, a}^{\dagger} w_{k, b}+w_{k, b}^{\dagger} w_{k, a}\right)$. We notice that $\comm{H_{int}(t)}{w_{k,a/b}} = 0$ unless $t_k<t <t_k +\Delta t$. The equation of motion of the operators $w_{k,a}$ and $w_{k,b}$ are thus:
\begin{equation}
    \frac{d w_{k,a/b}}{d t} = \begin{cases}
           - \frac{i}{\hbar}\comm{w_{k,b}}{\frac{\hbar V}{\Delta t} O_k} , & \text{if } t_k < t <t_k+\Delta t  \\
          0, & \text{otherwise}
\end{cases}
\end{equation}
Thus for all times $t<t_k$ where have $w_{k,a/b}(t) = w_{k,a/b}$, while after the time window of $t_k < t <t_k+\Delta t$ the evolution is given by the unitary operator:
\begin{equation}
    U(t_k,t_k+\Delta t) =\exp \left[-\frac{i}{\hbar} \int_{t_k}^{t_k+\Delta t} H_{\mathrm{int}}(t^\prime) d t^{\prime}\right] = \exp \left[-\frac{i}{\hbar} 
\Delta t \frac{\hbar V}{\Delta t} O_k \right] = \exp \left[-i V O_k \right] 
\end{equation}
we thus have that the operator $w_{k,a}(t_k + \Delta t)$ after the interaction (where the derivative is zero and it no longer changes) is:
\begin{equation}
    w_{k,a}(t_k + \Delta t) = U^\dagger(t_k,t_k+\Delta t) w_{k,a} U(t_k,t_k+\Delta t)  = \exp \left[i V O_k \right]  w_{k,a} \exp \left[-i V O_k \right] 
\end{equation}
We can then use the Baker-Hausdorf lemma \cite{Gerry2004IntroductoryOptics}: 
\begin{equation}
   \mathrm{e}^{i \lambda A} B \mathrm{e}^{-i \lambda A} = B + i \lambda [A,B] + \frac{(i \lambda)^2}{2!}[A,[A,B]] + \frac{(i \lambda)^3}{3!}[A,[A,[A,B]]] + ...
\end{equation}
together with the following commutators:
\begin{align}
    &\comm{O}{w_{k,a}} = \left[w_{k, a}^\dagger w_{k, b}+w_{k, b}^{\dagger} w_{k, a}, w_{k, a}\right] = \left[w_{k, a}^{\dagger} w_{k, b}, w_{k, a}\right]=\left[w_{k, a}^\dagger, w_{k, a}\right] w_{k, b} = -w_{k, b} \\
    & \comm{O}{w_{k,b}} = -w_{k, a}
\end{align}
which gives:
\begin{align}
     w_{k,a}(t_k + \Delta t) & = w_{k,a} - i V w_{k,b} + \frac{(i V)^2}{2 !} w_{k,a}-\frac{(i V)^3}{3!} w_{k,b}+\cdots \\
    & = w_{k,a}(1 - \frac{V^2}{2!} + \cdots ) + w_{k,b}(-iV + i \frac{V^3}{3!} + \cdots ) = \cos(V) w_{k,a} -i \sin(V) w_{k,b}
\end{align}
and similarly:
\begin{align}
    w_{k,b}(t_k + \Delta t) = \cos(V) w_{k,b} -i \sin(V) w_{k,a}
\end{align}

Notice that if we instead had chosen the interaction Hamiltonian: 
\begin{equation}
     H_{int}(t) = \sum_k f_k(t) \frac{\hbar i V}{\Delta t}\left(w_{k, b}^{\dagger} w_{k, a} - w_{k, a}^{\dagger} w_{k, b}\right) = \sum_k f_k(t)\frac{i \hbar V}{\Delta t} Q_k
\end{equation}
with $Q_K =  w_{k, b}^{\dagger} w_{k, a} - w_{k, a}^{\dagger} w_{k, b}$ and consequently $\comm{Q_k}{w_{k,a}} = - w_{k,b}$ and $\comm{Q_k}{w_{k,b}} = w_{k,b}$ , which would then give:
\begin{equation}
    w_{k,a}(t_k + \Delta t) = \exp \left[- V Q_k \right]  w_{k,a} \exp \left[V Q_k \right] = w_{k,a} + V - \frac{V^2}{2!}w_{k,a} - \frac{V^3}{3!}w_{k,b} = \cos(V) w_{k,a} + \sin(V) w_{k,b}   
\end{equation}
\begin{equation}
    w_{k,b}(t_k + \Delta t) =  \cos(V) w_{k,b} - \sin(V) w_{k,a}   
\end{equation}


\

We can derive similar input-output relations for the more general case where we interact with a quantum system with an arbitrary number of waveguides. We write the general Hamiltonian in eq. \eqref{eq:general_transformed} as: 
\begin{equation}
\begin{aligned}
H_k = H_{s} + H_{k,sb} + H_{k,b} \label{eq:generalinteraction}
\end{aligned}
\end{equation}
where $H_{s}$ is Hamiltonian of the quantum system Hamiltonian, $H_{k,sb} = \sum_{i=1}^N \hbar \sqrt{\gamma_i/\Delta t} \left(w_{k,i}^{\dagger} a+a^{\dagger} w_{k,i} \right)$ the interaction with the system, where $a$ is the annihilation of the quantum system (for a two-level system, $\sigma$ would be a more appropriate symbol), and $\gamma_i$ defines the interaction strength between the quantum system and the corresponding waveguide mode $i$. $ H_{k,b} = \sum_{i \neq j} \hbar  V_{i j}/\Delta t w_{k,i}^{\dagger} w_{k,j}$ defines the interaction between the waveguides modes $i$ and $j$. The time-evolution operator is now given as:
%Note that the form of eq. \eqref{eq:generalinteraction} requires $V_{\mu,\nu} = - V_{\nu,\mu}$ for there to be time-reversal symmetry. In addition, it is shown in ref. \cite{Xu2016FanoTransport} that a Hamiltonian of similar form also ensures flux conservation, energy conservation, and causality.
\begin{equation}
    U(t_k,t_k+\Delta t) =\exp \left[-\frac{i}{\hbar} \int_{t_k}^{t_k+\Delta t} [ H_s + H_{sb} + H_b ] d t^{\prime}\right] = \mathrm{e}^{ \left[ -i \Delta t / \hbar [ H_s + H_{sb} + H_b ] \right ]}
\end{equation}

This form is much more complicated and will, in general, lead to complex self-energy corrections, this is clear if we expand the above exponential using the Zassenhaus formula (in the following we omit a factor of $\Delta t/\hbar$ and absorb it into a normalized Hamiltonian $\Tilde{H} = \Delta t/\hbar H$):

\begin{align}
    U(t_k,t_k+\Delta t) &= \mathrm{e}^{ -i [ \Tilde{H}_s + \Tilde{H}_{sb} + \Tilde{H}_b ]} = \mathrm{e}^{ -i \Tilde{H}_b } \mathrm{e}^{ -i [ \Tilde{H}_{s} + \Tilde{H}_{sb}]} \mathrm{e}^{ 1/2 \comm{\Tilde{H}_b}{\Tilde{H}_{sb}}} \mathrm{e}^{ i/6  \comm{\Tilde{H}_b}{\comm{\Tilde{H}_b}{\Tilde{H}_{sb}}}} \mathrm{e}^{ -1/24 \comm{\comm{\comm{\Tilde{H}_{sb}}{\Tilde{H}_b}}{\Tilde{H}_b}}{\Tilde{H}_b}}\cdots \\
    &= \mathrm{e}^{ -i \Tilde{H}_b } \mathrm{e}^{ -i [ \Tilde{H}_{s} + \Tilde{H}_{sb}]} \mathrm{e}^{-i \Tilde{H}_{eff}} 
\end{align}
where it was used that $\comm{H_b}{H_s} = 0$ and also that $\comm{H_b}{H_{bs}}$ will only generate terms of the type $a w_{k,\mu}^\dagger$ and $a^\dagger w_{k,\mu}$ and so  $\comm{\comm{H_b}{H_{bs}}}{H_{bs}} = 0$. The infinite series will serve as a self-energy correction to the waveguide-system interaction $H_{eff}$, but the input-output relation will still just be given by $\mathrm{e}^{-\Tilde{H_b}}$. The input-output relation can be calculated from the commutator:
\begin{equation}
    \comm{w_{k,\mu}}{H_b} = \sum_{\nu \neq \mu} - i/\Delta t V_{\mu,\nu} w_{k,\nu}
\end{equation}
if we introduce the vector $\textbf{W} = \begin{pmatrix}
    w_{k,1} \\ w_{k,2} \\ \vdots \\ w_{k,N}
\end{pmatrix}$ where $N$ is the number of waveguides, and $\textbf{V} = \begin{pmatrix}
    0 & V_{1,2} & \cdots & V_{1,N} \\
    V_{2,1} & 0 & \cdots & V_{2,N} \\
     \vdots & \vdots & \ddots & \vdots \\
     V_{N,1} & V_{N,2} & \cdots & 0 \\
\end{pmatrix}$ we can write all commutator relations as:
\begin{equation}
    \comm{\textbf{W}}{H_b} = - i/\Delta t \textbf{V} \textbf{W}
\end{equation}
and the waveguide operators thus transform unitary evolution as:
\begin{equation}
    \mathrm{e}^{i \Delta t H_b} \textbf{W} \mathrm{e}^{-i \Delta t H_b} = \textbf{W} + i \Delta t\comm{H_b}{\textbf{W}} - \frac{\Delta t^2}{2!} \comm{H_b}{\comm{H_b}{\textbf{W}}} + \cdots = \textbf{W} + \textbf{V} \textbf{W} + \textbf{V}^2/2! \textbf{W} + \cdots = \exp(\textbf{V}) \textbf{W}  
\end{equation}
and we will thus have the total transformation:
\begin{align}
    U(t_k,t_k+\Delta t)^\dagger \textbf{W} U(t_k,t_k+\Delta t) &=  \mathrm{e}^{i \Tilde{H}_{eff}} \mathrm{e}^{ i [ \Tilde{H}_{s} + \Tilde{H}_{sb}]}  \mathrm{e}^{ i \Tilde{H}_b } \textbf{W} \mathrm{e}^{ -i \Tilde{H}_b } \mathrm{e}^{ -i [ \Tilde{H}_{s} + \Tilde{H}_{sb}]} \mathrm{e}^{-i \Tilde{H}_{eff}} \\
    &= \mathrm{e}^{-i \Tilde{H}_{eff}} \mathrm{e}^{ -i [ \Tilde{H}_{s} + \Tilde{H}_{sb}]}  \exp(\textbf{V}) \textbf{W} \mathrm{e}^{ -i [ \Tilde{H}_{s} + \Tilde{H}_{sb}]} \mathrm{e}^{-i \Tilde{H}_{eff}}
\end{align}
We thus see that if we did not have any local system, the waveguide operators would transform as $\textbf{W}_{out} = \textbf{W}(t+\Delta t) = \textbf{C} \textbf{W}(t)$ with $\textbf{C}=\exp(\textbf{V})$.






