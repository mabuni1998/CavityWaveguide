The ability to manipulate and control quantum states of light is vital in many quantum technology applications. Such applications include quantum cryptography and communication, where photons carry quantum information over long distances. But also quantum computing, where the light either mediates quantum information between localized qubits or computation is performed on the quantum state of light itself. An essential ingredient in manipulating such quantum states of light is the ability to make photons interact with each other. In the ultimate limit of quantum pulses containing one, two, or a few photons, the combination of low optical power and weak non-linearities in bulk materials has proven a significant challenge in achieving high-fidelity interactions \cite{Chang2014QuantumPhoton}. 

\

However, recent progress in the design and fabrication of nanostructures holds promise for efficient interaction with photons. Optical waveguides allow for guiding and manipulation of photons with the possibility for on-chip scalability \cite{Lodahl2015InterfacingNanostructures}. 
By integrating a localized emitter in the waveguide, the light-matter interaction can mediate photon-photon interactions leading to optical non-linearities \cite{Chang2014QuantumPhoton}. Indeed, an efficient light-matter interface in a photonic crystal waveguide with a semiconductor quantum dot has been demonstrated \cite{LeJeannic2022DynamicalEmitter}. Optical nonlinearities can also be enhanced further by optical cavities. For example, extreme confinement of light in dielectric cavities allows for unprecedented strong light-matter interactions \cite{Choi2017Self-SimilarNonlinearities,Wang2018MaximizingCavities,Hu2018ExperimentalResonators,Albrechtsen2022Nanometer-scaleCavities} and is a possible source of optical nonlinearities that are sensitive to single quanta of light. The Jaynes-Cummings model here aptly describes the interaction between the localized optical mode and quantum emitter \cite{Jaynes1963ComparisonMaser}, but the state of the emitted traveling photon is not described. When interfacing photons in waveguides, the state of the traveling photon, however, becomes relevant \cite{Sheremet2023WaveguideCorrelations}

\

Naturally, describing propagating states of light interacting with cavities or emitters has thus recently received a lot of attention, and the field of Waveguide Quantum ElectroDynamics (WQED) is an exciting new area of physics. The theory of traveling quantum states of light is inherently a rich and complex many-body problem. Much effort has been put into developing master equations \cite{Baragiola2012N-PhotonSystem} or input-output theory (SLH) \cite{Kiilerich2019Input-OutputPulses,Kiilerich2020QuantumRadiation,Lund2023PerfectPulse,Yang2022DeterministicSystems} that correctly predict the interaction of wavepackets with cavity-emitter systems. In these descriptions, the complete state of the traveling wavepacket is  not described, and treating non-markovian effects, such as delayed feedback, is hard and requires substantial additions to the calculations. To treat such problems, matrix product states can be used to describe the complete state of the waveguide in an efficient way \cite{ArranzRegidor2021ModelingModel,Richter2022EnhancedNetworks}. Matrix product states are, however, complex and offer little physical insight into the system. Therefore, instead, more simple and intuitive space-discretized waveguide models have been proposed \cite{Crowder2020QuantumFeedback,Crowder2022QuantumFeedback}.  

\

In this master thesis, we take a similar approach to the space-discretized waveguide models, where we use a quantum collision model \cite{Ciccarello2018CollisionOptics} to describe the traveling wavepacket as a collection of time-bins interacting with the quantum system one at a time \cite{Heuck2020Photon-photonCavities,Heuck2020Controlled-PhaseNonlinearities,Krastanov2022Controlled-phaseEmitter}. We will present this model as a numerical framework that is conceptually simple and easy to use for researchers familiar with quantum optics simulation softwares such as QuTiP in python \cite{Johansson2012QuTiP:Systems,Johansson2013QuTiPSystems}, Quantum Toolbox in Matlab, and QuantumOptics.jl in Julia \cite{Kramer2018QuantumOptics.jl:Systems}.  Code and documentation of our numerical framework WaveguideQED.jl is available at: \hyperlink{https://github.com/qojulia/WaveguideQED.jl}{https://github.com/qojulia/WaveguideQED.jl} 

\

Using the framework, we investigate the scattering of single and two-photon pulses on cavities and emitters coupled to single and multiple channels. We also allow for direct interactions between the waveguide channels and show how this affects the emission rate and frequency due to a change in the Local Density of Optical states. Such coupling is not conventionally included in other discretized waveguide models, and we here discuss the difficulties in doing so. Finally, we also show that our framework is capable of modeling non-Markovian dynamics arising from delayed feedback with little added complexity. As an example, we consider a semi-infinite waveguide with a mirror in one end leading to a feedback loop and, for some conditions, excitation trapping.



%Quantum information technologu
%Photons are excellent for carrying quantum information over long distances due to their weak interaction with their surroundings. This robustness is vital in quantum communication, quantum networks, distributed quantum computation, and quantum photonic computation. Common for all such applications is the need to have photon-photon interactions either in the form of a beamsplitter followed by detection or some non-linear medium.   

%to distribute or establish entanglement across stationary qubits. Optical cavities are here essential to enhance the interaction between, e.g., a quantum dot or NV-color center and the optical mode. While the theory of a single cavity mode interacting with an emitter is well understood through the Jaynes-Cummings model, the description of a traveling wave packet containing quantum states of light is potentially a complex many-body problem. Naturally, much effort has been put into developing master equations \cite{Baragiola2012N-PhotonSystem} or other approaches \cite{Kiilerich2019Input-OutputPulses,Kiilerich2020QuantumRadiation} that correctly predict the interaction of such wavepackets with cavity-emitter systems. However, in such descriptions, the complete state of the traveling wavepacket, which is relevant for precise fidelity calculations, is not described. In this project, we will develop a numerical tool to describe the traveling wavepacket state based on the theory of photon-time binning \cite{Heuck2020Photon-photonCavities,Heuck2020Controlled-PhaseNonlinearities,Krastanov2022Controlled-phaseEmitter}.